\section{Preliminaries for finite deformation hyperelastic formulation}\label{sec:preliminaries}

%----------------
% Kinematic
%----------------
\subsection{Kinematic preliminaries}
To set the stage for the hyperelastic formulation, the kinematic
preliminaries for large deformation elastoplasticity are summarized 
in this section, along with the quantities and relations that will be
used in subsequent model development.

An essential feature of this elastoplasticity framework is the
multiplicative decomposition of the deformation gradient $\bF$ into 
an elastic part $\bF^e$ and a plastic part $\bF^p$

\begin{equation}
  \bF = \bF^e \cdot \bF^p.
\end{equation}

This decomposition introduces the notion of an intermediate local
configuration (\cf\ \cite{SimoHughes:98} and the references therein
for the motivation and micromechanical basis for such a
decomposition).

Next, we introduce a set of strain measures associated with the
multiplicative decomposition that will be used extensively in the
model development. First is the right Cauchy-Green tensor $\bC$, and
its plastic counterpart $\bC^p$, which are defined in the reference
configuration

\begin{align}
  \bC &= \bF^T \cdot \bF\\ \bC^p &= \bF^{pT} \cdot \bF^p
\end{align}
where $\bF^T$ is the transpose of $\bF$.

In the current configuration we consider the left Cauchy-Green tensor
$\bb$, and its elastic counterpart ${\bb}^e$

\begin{align}
  \bb &= \bF \cdot \bF^{T}\\ \bb^e &= \bF^e \cdot
  \bF^{eT}\label{eq:be}
\end{align}

The above fundamental strain measures are related via pull-back and
push-forward operations

\begin{align}\label{eq:pullpush}
  \bC^{-p} &=\bF^{-1} \cdot \bb^e \cdot \bF^{-T}\\ \bb^e &= \bF \cdot
  \bC^{-p} \cdot \bF^T
\end{align}

In metal plasticity, a standard assumption is that plastic flow is
isochoric (volume-preserving), \ie\ $\det(\bF^p)=1$, which implies

\begin{equation}
  J = \det (\bF) = \det (\bF^e)
\end{equation}

Defining $J^e=\det(\bF^e)$, we have the volume preserving part of the
elastic left Cauchy-Green tensor $\bar{\bb}^e$

\begin{equation}
  \bar{\bb}^e = J^{e^{-2/3}}\bb^e = J^{-2/3}\bb^e.
\end{equation}

%----------------
% Hyperelastic
%----------------
\subsection{Hyperelastic constitutive relation}
The starting point of the hyperelastic constitutive formulation is 
the assumption of the existence of a strain-energy function, which is
proposed to have the following form

\begin{equation}
\Psi=\Psi^{\rm vol}[J^e] + \Psi^{\rm iso}[\bar{\bb}^e]
\end{equation}

Here the strain-energy function $\Psi$ is a decoupled function of the
volumetric part (i.e., $J^e=\det\bF^e$) and the isochoric part (i.e.,
$\bar{\bb}^e$) of the elastic deformation. The volumetric and the
isochoric parts of the strain-energy function are given as

\begin{equation}
  \Psi^{\rm vol}[J^e] =\frac{\kappa}{4}\left(J^e{^2} - 1 - \rm ln~J^e{^2}\right)
\end{equation}

\begin{equation}
  \Psi^{\rm iso}[\bar{\bb}^e] =
  \frac{\mu}{2}\left(\rm tr(\bar{\bb}^e)-3\right)
\end{equation}

where $\kappa$ and $\mu$ are the bulk and shear modulus. Here we are
following the developments of (\cite{Steinmann1994}), however we
employ a different choice of strain energy-functions. The elastic
constitutive law and the Kirchhoff stresses are given as

\begin{equation}\label{eq:Kirchhoff}
\btau=J^e p \bone + \bs
\end{equation}

The Kirchhoff pressure $p$ and the deviatoric stress tensor $\bs$ are
related to the elastic strain measure as

\begin{equation}\label{eq:p}
p = \frac{1}{3}\tr(\btau) = \frac{\kappa}{2}\left(J^e{^2}-1\right)/J^e
\end{equation}

\begin{equation}\label{eq:s}
\bs = \dev(\btau) = \mu\dev(\bar{\bb}^e)
\end{equation}


%%%%%%%%%%%%%%%%%%%%%%%%%%%%%%%%%%%%%%%%%%
% FORMULATION
%%%%%%%%%%%%%%%%%%%%%%%%%%%%%%%%%%%%%%%%%%
\section{Formulation of the finite deformation Gurson model} \label{sec:formulation}

Within the previously described large deformation hyperelastic
framework, key components of the constitutive relations of the
shear-modified Gurson model are presented in this section, including
the yield function, the hardening law, the flow rule and the 
evolution law for the shear-modified void growth.

%----------------
% Yield function
%----------------
\subsection{Yield function}
The yield function $\Phi$ of the Gurson model can be written in terms
of the previously defined Kirchhoff mean stress $p$ and deviatoric
stress tensor $\bs$ as

\begin{equation}\label{eq:Phi}
\Phi = \frac{1}{2}\bs:\bs - \frac{1}{3}\psi Y^2
\end{equation}

where $\psi$ contains contributions from the damage in the material
and $Y$ is the Kirchhoff yield stress. This yield function
\eqref{eq:Phi} is a quadratic version form. This particular variant 
of the yield function was chosen as it proved to exhibit superior
properties of robustness in convergence in the nonlinear solution
algorithm outline below.

The function $\psi$ directly relates to the void volume fraction of
the porous solid and is given as

\begin{equation}
\psi = 1 + q_3 f^2 - 2 q_1 f \cosh(v), ~~v=\frac{3 q_2 p}{2 Y}
\end{equation}

where $p$ is the Kirchhoff pressure defined in \eqref{eq:p}, and $f$
is the void volume fraction of the porous solid. The Kirchhoff yield
stress $Y$ describes the hardening of the undamaged matrix
material. Also, the constants $q_1,q_22,q_3$ are the accepted
additional yield parameters often present in GTN type models.

%----------------
% Hardening law
%----------------
\subsection{Hardening law}
The hardening law relates the yield strength $Y$ to some measure of
plastic deformation. One example of a nonlinear hardening law proposed
by \cite{SimoHughes:98} for metal is written as
\begin{equation}
Y = Y_0 + Y_{\infty}\left[ 1-\exp(-\delta\varepsilon_q)\right] +
K\varepsilon_q
\end{equation}

where $\varepsilon_q$ is the equivalent plastic strain, $Y_0$ is the
initial yield strength, $Y_{\infty}$ is the residual flow stress, $K$
is the hardening coefficient, and $\delta$ is the saturation
exponent. Other forms of hardening law can also be used depending on
the observed material behavior.

The plastic work increment in the matrix material is equal to the
macroscopic plastic work increment, which can be used to derive the
evolution equation for the equivalent plastic strain $\varepsilon_q$
as

\begin{equation}\label{eq:doteq_1}
\dot{\varepsilon}_q Y(1-f) = \btau : \left(
\gamma\frac{\partial\Phi}{\partial\btau} \right)
\end{equation}

Substituting yield function $\Phi$ into \eqref{eq:doteq_1} to obtain
the expression to compute $\dot{\varepsilon}_q$
\begin{equation}\label{eq:doteq}
  \dot{\varepsilon}_q =\frac{1}{1-f}\left(
  \gamma\sqrt{\frac{2|\psi|}{3}} {\rm sign}(\psi) + \frac{p t}{Y}
  \right)
\end{equation}


%----------------
% Flow rule
%----------------
\subsection{Flow rule}
Following the standard procedure of the principle of maximum
dissipation, Simo and Miehe \cite{Simo1992} proposed a general form 
of associate flow rule, which is adopted in the current formulation 
and is given as

\begin{equation}\label{eq:flowRule}
-\frac{1}{2}L_v(\bb^e)\cdot\bb^{-e} =
\gamma\frac{\partial\Phi}{\partial\btau}=\gamma\bn +
\frac{1}{3}t\tensor 1
\end{equation}

where $L_v(\bb^e)=\bF \cdot \dot{\bC}^{-p} \cdot \bF^T$ is the Lie
derivative, and $\bn$ and $t$ are the deviatoric and volumetric
component of the gradient term, respectively. Substituting the yield
function \eqref{eq:Phi} leads to the following

\begin{equation}\label{eq:n}
\bn = \frac{\bs}{\|\bs\|}
\end{equation}

\begin{equation}\label{eq:t}
t = \tr \left( \gamma \frac{\partial\Phi}{\partial\btau}\right) =
\sqrt{\frac{3}{2}}\gamma f \sinh (\frac{3p}{2
  Y})\frac{1}{\sqrt{|\psi|}} {\rm sign}(\psi)
\end{equation}


%----------------
% Void volume
%----------------
\subsection{Evolution of void volume fraction and shear modification}
The void volume fraction $f$ is the internal variable that
characterizes the material damage. The rate of change in total void
volume fraction, $\dot{f}$, is typically given by the sum of
contributions due to the void growth, $\dot{f}_g$, and the nucleation
of new voids, $\dot{f}_{n}$.

\begin{equation}\label{eq:dotf}
\dot{f} = \dot{f}_g + \dot{f}_{n}\\
\end{equation}

In the original Gurson model \cite{Gurson1977}, the void growth part
$\dot{f}_g$ was related to the plastic volume change as

\begin{equation}\label{eq:fg_orig}
\dot{f}_g = (1-f)\tr\left( \gamma\frac{\partial\Phi}{\partial\btau}
\right)
\end{equation}

To account for the void growth under shear-dominated stress state, 
the void growth law \eqref{eq:fg_orig} was extended in
\cite{Nahshon2008} by adding a term that depends on the third stress 
invariant. This shear-modified void growth equation is written as

\begin{equation}\label{eq:fg_1}
\dot{f}_g = (1-f)\tr\left( \gamma\frac{\partial\Phi}{\partial\btau}
\right) + k_{\omega}f\frac{\omega(\btau)}{\tau_e}\bs: \left(
\gamma\frac{\partial\Phi}{\partial\btau} \right)
\end{equation}

where $\tau_e=\sqrt{3/2}\|\bs\|$ is the effective deviatoric 
Kirchhoff stress, $k_{\omega}$ is a material constant that sets the 
magnitude of the damage growth rate in pure shear states 
\cite{Nahshon2008}. The function $\omega(\btau)$ includes the effect 
of the third stress invariant on void growth and is given as

\begin{equation}\label{eq:omega}
\omega(\btau) = 1- \left(\frac{27J_3}{2\tau_e^3}\right)^2
\end{equation}
where $J_3=\det(\bs)$ is the third invariant of deviatoric Kirchhoff
stress tensor.

Substituting the yield function \eqref{eq:Phi} and the expression for
$\omega(\btau)$ \eqref{eq:omega} into the void growth law
\eqref{eq:fg_1} yields

\begin{equation}\label{eq:fg_final}
\dot{f}_g = (1-f)t + \sqrt{\frac{2}{3}}\gamma k_{\omega}f
\omega(\btau)
\end{equation}

The effective increase in damage due to plastic strain controlled
nucleation is given by \cite{Chu1980} as

\begin{equation}\label{eq:dotf_nu}
\dot{f}_{nu} = A \dot{\varepsilon}_q
\end{equation}

where the parameter A is defined as a function of the matrix
equivalent plastic strain $\varepsilon_q$

\begin{equation}
A =
   \begin{cases}
     \frac{f_N}{s_N\sqrt{2\pi}}\exp\left[ -\frac{1}{2}\left(
       \frac{\varepsilon_q - \epsilon_N}{s_N}\right)^2\right], & p
     \geq 0\\ 0, & p <0
	\end{cases}
\end{equation}

where the nucleation strain follows a normal distribution with a mean
value $\epsilon_N$ and a standard deviation $s_N$ with the volume
fraction of the nucleated voids given by $f_N$.


% Local Variables:
% TeX-master: "GursonModel"
% mode: latex
% mode: flyspell
% End:
