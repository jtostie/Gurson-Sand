
\chapter{Introduction}
\label{intro}

The purpose of this work is to present a modified Gurson constitutive
model used in capturing the behavior of ductile materials up to and
including the failure regime. The original Gurson model was developed
by \cite{Gurson1977} based on rigorous micromechanical analysis of a
characteristic volume element with spherical or cylindrical-shaped
voids surrounded by rigid plastic matrix material. This model was
motivated by experimental observations of dilational plastic
deformation during ductile fracture in porous metals, which generate
considerable porosity due to the nucleation and growth of
voids. Tvergaard and Needleman \cite{Tvergaard1984} later improved the
original Gurson model by introducing additional parameters into the
yield function and by introducing an effective void volume fraction
term to account for coalescence of voids that more accurately captured
the void volume fraction growth rates. This model is referred to as
the 'Gurson-Tvergaard-Needleman' or the GTN model.

One important limitation of the original Gurson or the GTN model is
that the void growth (i.e., the damage) depends only on the mean
stress. In a shear-dominated state, such as in a projectile
penetration problem, the model is unable to predict damage growth if
continuous void nucleation is not invoked. This limitation motivates a
modification of the original void growth law to include a shear term
proposed in \cite{Nahshon2008}. This modification, though
phenomenological in nature, has been shown to improve the prediction
of the Gurson model in situations where shear stress dominates
\cite{Nahshon2008,Nahshon2009}.

The original Gurson model and the recently introduced shear
modification were both formulated either with a small deformation
assumption or within a hypoelastic framework. The large deformation
setting typically encountered in a ductile failure simulation would
render the small deformation formulation inappropriate. As for the
hypoelastic formulation, there are well known drawbacks such as the
non-zero work done in a closed cycle of elastic deformation, which
violates the most important axiom of an elastic response
\cite{Belytschko2013}.

To avoid the above problems, in Section~\ref{model-form}, the
shear-modified Gurson model in \cite{Nahshon2008, Nahshon2009} will be
reformulated within a large deformation hyperelastic constitutive
framework. Since the elastic response is derived from a hyperelastic
potential, the work done in a closed elastic deformation loop vanishes
exactly. Furthermore, the hyperelastic formulation eliminates the need
for incrementally objective stress update algorithms and can be easily
integrated with frame-invariant formulations of anisotropic elasticity
and anisotropic plastic yielding \cite{Belytschko2013}. The scope of
the current work, however, is limited to an isotropic treatment.

The remainder of this report includes a discussion of the model
formulation in Chapter~\ref{model-form} and a discussion of the
implementation in
Chapter~\ref{implementation}. Chapter~\ref{numerical-examples}
showcases a select few numerical examples of the
model. Chapter~\ref{conclusions} finishes with a conclusions and a
discussion of the efficacy of the model.
% Local Variables:
% TeX-master: "GursonModel"
% mode: latex
% mode: flyspell
% End:
