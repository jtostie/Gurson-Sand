\documentclass[11pt]{article}

\usepackage{fancyhdr}
\usepackage{fullpage}
\usepackage{amsmath}
\usepackage{amsbsy}
\usepackage{amssymb}
\usepackage{amscd}
\usepackage{amsfonts}
\usepackage{amsthm}
\usepackage{supertabular}
\usepackage{graphicx}
\usepackage{verbatim}
\usepackage{epsfig}
\usepackage{xspace}
\usepackage{euscript}
\usepackage{alltt}
\usepackage{boxedminipage}
\usepackage{float}
\usepackage{times}
\usepackage{epic}
\usepackage{eepic}
\usepackage{ifthen}
\usepackage{algorithm}
\usepackage{booktabs}
\usepackage{multirow}
\usepackage{cancel}
%\usepackage[colorlinks]{hyperref}
\usepackage[numbers,sort&compress]{natbib}
\usepackage[FIGBOTCAP,TABTOPCAP,bf,tight]{subfigure}
\usepackage{tabularx}
\usepackage[T1]{fontenc}
\usepackage{ae,aecompl}

% added:
\theoremstyle{remark}
\newtheorem{rmk}{Remark}

\makeatletter \@addtoreset{figure}{section}
\def\thefigure{\thesection.\@arabic\c@figure} \def\fps@figure{h, t}
\@addtoreset{equation}{section}
\def\theequation{\thesection.\arabic{equation}} \makeatother

%%%%%%%%%%%%%%%%%%%%%%%%%%%%%%%%%%%%%%%%%%%%%%%%%%%%%%%%%%%%%%%%%%%%%%%%
% editing
%%%%%%%%%%%%%%%%%%%%%%%%%%%%%%%%%%%%%%%%%%%%%%%%%%%%%%%%%%%%%%%%%%%%%%%%
%\usepackage[usenames,dvipsnames]{xcolor}
%\newcommand{\comment}[1]{\textcolor{blue}{[ \sc{#1} ]}} % comments
%\newcommand{\revise}[1]{\textcolor{red}{{#1}}} % revisions

%%%%%%%%%%%%%%%%%%%%%%%%%%%%%%%%%%%%%%%%%%%%%%%%%%%%%%%%%%%%%%%%%%%%%%%%
% abbreviations
%%%%%%%%%%%%%%%%%%%%%%%%%%%%%%%%%%%%%%%%%%%%%%%%%%%%%%%%%%%%%%%%%%%%%%%%
\newcommand{\aref}[1]{{App.~\ref{#1}}}
\newcommand{\eref}[1]{{Eq.~(\ref{#1})}}
\newcommand{\cref}[1]{{Ref.~\cite{#1}}}
\newcommand{\sref}[1]{{Sec.~\ref{#1}}}
\newcommand{\fref}[1]{{Fig.~\ref{#1}}}
\newcommand{\tref}[1]{{Table~\ref{#1}}}
\newcommand{\cf}{{cf.\,}}
\newcommand{\ie}{{i.e.\,}}
\newcommand{\vs}{{vs.\,}}
\newcommand{\eg}{{e.g.\,}}
\newcommand{\etal}{{et al.\,}}

%%%%%%%%%%%%%%%%%%%%%%%%%%%%%%%%%%%%%%%%%%%%%%%%%%%%%%%%%%%%%%%%%%%%%%%%
% tensors
%%%%%%%%%%%%%%%%%%%%%%%%%%%%%%%%%%%%%%%%%%%%%%%%%%%%%%%%%%%%%%%%%%%%%%%%
\renewcommand{\vec}[1]{{\boldsymbol{#1}}}
\newcommand{\chib}{\vec{\chi}}

%% suffix
\newcommand{\ab}{\vec{a}}
%\newcommand{\bb}{\vec{b}}
\newcommand{\vb}{\vec{v}}
\newcommand{\Ab}{\vec{A}}
\newcommand{\Bb}{\vec{B}}
\newcommand{\Tb}{\vec{T}}

%% prefix
\newcommand{\bzero}{\vec{0}}
\newcommand{\bone}{\vec{1}}

\newcommand{\bA}{\vec{A}}
\newcommand{\bB}{\vec{B}}
\newcommand{\bC}{\vec{C}}
\newcommand{\bD}{\vec{D}}
\newcommand{\bE}{\vec{E}}
\newcommand{\bF}{\vec{F}}
\newcommand{\bG}{\vec{G}}
\newcommand{\bH}{\vec{H}}
\newcommand{\bI}{\vec{I}}
\newcommand{\bJ}{\vec{J}}
\newcommand{\bK}{\vec{K}}
\newcommand{\bL}{\vec{L}}
\newcommand{\bM}{\vec{M}}
\newcommand{\bN}{\vec{N}}
\newcommand{\bO}{\vec{O}}
\newcommand{\bP}{\vec{P}}
\newcommand{\bQ}{\vec{Q}}
\newcommand{\bR}{\vec{R}}
\newcommand{\bS}{\vec{S}}
\newcommand{\bT}{\vec{T}}
\newcommand{\bU}{\vec{U}}
\newcommand{\bV}{\vec{V}}
\newcommand{\bW}{\vec{W}}
\newcommand{\bX}{\vec{X}}
\newcommand{\bY}{\vec{Y}}
\newcommand{\bZ}{\vec{Z}}

\newcommand{\ba}{\vec{a}}
\newcommand{\bb}{\vec{b}}
\newcommand{\bc}{\vec{c}}
\newcommand{\bd}{\vec{d}}
\newcommand{\be}{\vec{e}}
%\renewcommand{\bf}{\vec{f}}
\newcommand{\bff}{\vec{f}}
\newcommand{\bg}{\vec{g}}
\newcommand{\bh}{\vec{h}}
\newcommand{\bi}{\vec{i}}
\newcommand{\bj}{\vec{j}}
\newcommand{\bk}{\vec{k}}
\newcommand{\bl}{\vec{l}}
\newcommand{\bm}{\vec{m}}
\newcommand{\bn}{\vec{n}}
\newcommand{\bo}{\vec{o}}
\newcommand{\bp}{\vec{p}}
\newcommand{\bq}{\vec{q}}
\newcommand{\br}{\vec{r}}
\newcommand{\bs}{\vec{s}}
\newcommand{\bt}{\vec{t}}
\newcommand{\bu}{\vec{u}}
\newcommand{\bv}{\vec{v}}
\newcommand{\bw}{\vec{w}}
\newcommand{\bx}{\vec{x}}
\newcommand{\by}{\vec{y}}
\newcommand{\bz}{\vec{z}}

\newcommand{\balpha}{\vec{\alpha}}
\newcommand{\bbeta}{\vec{\beta}}
\newcommand{\bgamma}{\vec{\gamma}}
\newcommand{\bdelta}{\vec{\delta}}
\newcommand{\bepsilon}{\vec{\epsilon}}
\newcommand{\bvarepsilon}{\vec{\varepsilon}}
\newcommand{\bzeta}{\vec{\zeta}}
\newcommand{\beeta}{\vec{\eta}}
\newcommand{\btheta}{\vec{\theta}}
\newcommand{\bvartheta}{\vec{\vartheta}}
\newcommand{\bkappa}{\vec{\kappa}}
\newcommand{\blambda}{\vec{\lambda}}
\newcommand{\bmu}{\vec{\mu}}
\newcommand{\bnu}{\vec{\nu}}
\newcommand{\bxi}{\vec{\xi}}
\newcommand{\bomicron}{\vec{o}}
\newcommand{\bpi}{\vec{\pi}}
\newcommand{\bvarpi}{\vec{\varpi}}
\newcommand{\brho}{\vec{\rho}}
\newcommand{\bvarrho}{\vec{\varrho}}
\newcommand{\bsigma}{\vec{\sigma}}
\newcommand{\bvarsigma}{\vec{\varsigma}}
\newcommand{\btau}{\vec{\tau}}
\newcommand{\bupsilon}{\vec{\upsilon}}
\newcommand{\bphi}{\vec{\phi}}
\newcommand{\bvarphi}{\vec{\varphi}}
\newcommand{\bchi}{\vec{\chi}}
\newcommand{\bpsi}{\vec{\psi}}
\newcommand{\bomega}{\vec{\omega}}

\newcommand{\bGamma}{\vec{\Gamma}}
\newcommand{\bDelta}{\vec{\Delta}}
\newcommand{\bTheta}{\vec{\Theta}}
\newcommand{\bLambda}{\vec{\Lambda}}
\newcommand{\bXi}{\vec{\Xi}}
\newcommand{\bPi}{\vec{\Pi}}
\newcommand{\bSigma}{\vec{\Sigma}}
\newcommand{\bUpsilon}{\vec{\Upsilon}}
\newcommand{\bPhi}{\vec{\Phi}}
\newcommand{\bPsi}{\vec{\Psi}}
\newcommand{\bOmega}{\vec{\Omega}}

%%%%%%%%%%%%%%%%%%%%%%%%%%%%%%%%%%%%%%%%%%%%%%%%%%%%%%%%%%%%%%%%%%%%%%%%
% matrix
%%%%%%%%%%%%%%%%%%%%%%%%%%%%%%%%%%%%%%%%%%%%%%%%%%%%%%%%%%%%%%%%%%%%%%%%
\newcommand{\mat}[1]{\mathsf{#1}}
\newcommand{\transpose}[1]{{#1}^{\text{T}}}
\newcommand{\conjtransla}[1]{{#1}^{\text{*}}}
\newcommand{\conjtransqm}[1]{{#1}^{\dagger}}

%% suffix
\newcommand{\ws}{\mat{w}}
\newcommand{\As}{\mat{A}}
\newcommand{\Ms}{\mat{M}}

%% prefix
\newcommand{\szero}{\mat{0}}
\newcommand{\sone}{\mat{1}}

\newcommand{\sA}{\mat{A}}
\newcommand{\sB}{\mat{B}}
\newcommand{\sC}{\mat{C}}
\newcommand{\sD}{\mat{D}}
\newcommand{\sE}{\mat{E}}
\newcommand{\sF}{\mat{F}}
\newcommand{\sG}{\mat{G}}
\newcommand{\sH}{\mat{H}}
\newcommand{\sI}{\mat{I}}
\newcommand{\sJ}{\mat{J}}
\newcommand{\sK}{\mat{K}}
\newcommand{\sL}{\mat{L}}
\newcommand{\sM}{\mat{M}}
\newcommand{\sN}{\mat{N}}
\newcommand{\sO}{\mat{O}}
\newcommand{\sP}{\mat{P}}
\newcommand{\sQ}{\mat{Q}}
\newcommand{\sR}{\mat{R}}
\newcommand{\sS}{\mat{S}}
\newcommand{\sT}{\mat{T}}
\newcommand{\sU}{\mat{U}}
\newcommand{\sV}{\mat{V}}
\newcommand{\sW}{\mat{W}}
\newcommand{\sX}{\mat{X}}
\newcommand{\sY}{\mat{Y}}
\newcommand{\sZ}{\mat{Z}}

\newcommand{\sa}{\mat{a}}
\renewcommand{\sb}{\mat{b}}
\renewcommand{\sc}{\mat{c}}
\newcommand{\sd}{\mat{d}}
\newcommand{\se}{\mat{e}}
\renewcommand{\sf}{\mat{f}}
\newcommand{\sg}{\mat{g}}
\newcommand{\sh}{\mat{h}}
\newcommand{\si}{\mat{i}}
\newcommand{\sj}{\mat{j}}
\newcommand{\sk}{\mat{k}}
\renewcommand{\sl}{\mat{l}}
\newcommand{\sm}{\mat{m}}
\newcommand{\sn}{\mat{n}}
\newcommand{\so}{\mat{o}}
\renewcommand{\sp}{\mat{p}}
\newcommand{\sq}{\mat{q}}
\newcommand{\sr}{\mat{r}}
\renewcommand{\ss}{\mat{s}}
\newcommand{\st}{\mat{t}}
\newcommand{\su}{\mat{u}}
\newcommand{\sv}{\mat{v}}
\newcommand{\sw}{\mat{w}}
\newcommand{\sx}{\mat{x}}
\newcommand{\sy}{\mat{y}}
\newcommand{\sz}{\mat{z}}

\newcommand{\salpha}{\mat{\alpha}}
\newcommand{\sbeta}{\mat{\seta}}
\newcommand{\sgamma}{\mat{\gamma}}
\newcommand{\sdelta}{\mat{\delta}}
\newcommand{\sepsilon}{\mat{\epsilon}}
\newcommand{\svarepsilon}{\mat{\varepsilon}}
\newcommand{\szeta}{\mat{\zeta}}
\newcommand{\seta}{\mat{\eta}}
\newcommand{\stheta}{\mat{\theta}}
\newcommand{\svartheta}{\mat{\vartheta}}
\newcommand{\skappa}{\mat{\kappa}}
\newcommand{\slambda}{\mat{\lambda}}
\newcommand{\smu}{\mat{\mu}}
\newcommand{\snu}{\mat{\nu}}
\newcommand{\sxi}{\mat{\xi}}
\newcommand{\somicron}{\mat{o}}
\newcommand{\spi}{\mat{\pi}}
\newcommand{\svarpi}{\mat{\varpi}}
\newcommand{\srho}{\mat{\rho}}
\newcommand{\svarrho}{\mat{\varrho}}
\newcommand{\ssigma}{\mat{\sigma}}
\newcommand{\svarsigma}{\mat{\varsigma}}
\newcommand{\stau}{\mat{\tau}}
\newcommand{\supsilon}{\mat{\upsilon}}
\newcommand{\sphi}{\mat{\phi}}
\newcommand{\svarphi}{\mat{\varphi}}
\newcommand{\schi}{\mat{\chi}}
\newcommand{\spsi}{\mat{\psi}}
\newcommand{\somega}{\mat{\omega}}

\newcommand{\sGamma}{\mat{\Gamma}}
\newcommand{\sDelta}{\mat{\Delta}}
\newcommand{\sTheta}{\mat{\Theta}}
\newcommand{\sLambda}{\mat{\Lambda}}
\newcommand{\sXi}{\mat{\Xi}}
\newcommand{\sPi}{\mat{\Pi}}
\newcommand{\sSigma}{\mat{\Sigma}}
\newcommand{\sUpsilon}{\mat{\Upsilon}}
\newcommand{\sPhi}{\mat{\Phi}}
\newcommand{\sPsi}{\mat{\Psi}}
\newcommand{\sOmega}{\mat{\Omega}}

%%%%%%%%%%%%%%%%%%%%%%%%%%%%%%%%%%%%%%%%%%%%%%%%%%%%%%%%%%%%%%%%%%%%%%%%
% fourth-order tensors, common number sets
% latin/roman symbols only for now
%%%%%%%%%%%%%%%%%%%%%%%%%%%%%%%%%%%%%%%%%%%%%%%%%%%%%%%%%%%%%%%%%%%%%%%%

%% suffix

%% prefix
\newcommand{\bbA}{\mathbb{A}}
\newcommand{\bbB}{\mathbb{B}}
\newcommand{\bbC}{\mathbb{C}}
\newcommand{\bbD}{\mathbb{D}}
\newcommand{\bbE}{\mathbb{E}}
\newcommand{\bbF}{\mathbb{F}}
\newcommand{\bbG}{\mathbb{G}}
\newcommand{\bbH}{\mathbb{H}}
\newcommand{\bbI}{\mathbb{I}}
\newcommand{\bbJ}{\mathbb{J}}
\newcommand{\bbK}{\mathbb{K}}
\newcommand{\bbL}{\mathbb{L}}
\newcommand{\bbM}{\mathbb{M}}
\newcommand{\bbN}{\mathbb{N}}
\newcommand{\bbO}{\mathbb{O}}
\newcommand{\bbP}{\mathbb{P}}
\newcommand{\bbQ}{\mathbb{Q}}
\newcommand{\bbR}{\mathbb{R}}
\newcommand{\bbS}{\mathbb{S}}
\newcommand{\bbT}{\mathbb{T}}
\newcommand{\bbU}{\mathbb{U}}
\newcommand{\bbV}{\mathbb{V}}
\newcommand{\bbW}{\mathbb{W}}
\newcommand{\bbX}{\mathbb{X}}
\newcommand{\bbY}{\mathbb{Y}}
\newcommand{\bbZ}{\mathbb{Z}}

\newcommand{\bba}{\mathbb{a}}
\newcommand{\bbb}{\mathbb{b}}
\newcommand{\bbc}{\mathbb{c}}
\newcommand{\bbd}{\mathbb{d}}
\newcommand{\bbe}{\mathbb{e}}
\newcommand{\fb}{\mathbb{f}}
\newcommand{\bbg}{\mathbb{g}}
\newcommand{\bbh}{\mathbb{h}}
\newcommand{\bbi}{\mathbb{i}}
\newcommand{\bbj}{\mathbb{j}}
\newcommand{\bbk}{\mathbb{k}}
\newcommand{\bbl}{\mathbb{l}}
\newcommand{\bbm}{\mathbb{m}}
\newcommand{\bbn}{\mathbb{n}}
\newcommand{\bbo}{\mathbb{o}}
\newcommand{\bbp}{\mathbb{p}}
\newcommand{\bbq}{\mathbb{q}}
\newcommand{\bbr}{\mathbb{r}}
\newcommand{\bbs}{\mathbb{s}}
\newcommand{\bbt}{\mathbb{t}}
\newcommand{\bbu}{\mathbb{u}}
\newcommand{\bbv}{\mathbb{v}}
\newcommand{\bbw}{\mathbb{w}}
\newcommand{\bbx}{\mathbb{x}}
\newcommand{\bby}{\mathbb{y}}
\newcommand{\bbz}{\mathbb{z}}

%%%%%%%%%%%%%%%%%%%%%%%%%%%%%%%%%%%%%%%%%%%%%%%%%%%%%%%%%%%%%%%%%%%%%%%%
% third order tensors
%%%%%%%%%%%%%%%%%%%%%%%%%%%%%%%%%%%%%%%%%%%%%%%%%%%%%%%%%%%%%%%%%%%%%%%%
\newcommand{\cb}[1]{\boldsymbol{\mathcal{#1}}}

%% prefix
\newcommand{\cbA}{\cb{A}}
\newcommand{\cbB}{\cb{B}}
\newcommand{\cbC}{\cb{C}}
\newcommand{\cbD}{\cb{D}}
\newcommand{\cbE}{\cb{E}}
\newcommand{\cbF}{\cb{F}}
\newcommand{\cbG}{\cb{G}}
\newcommand{\cbH}{\cb{H}}
\newcommand{\cbI}{\cb{I}}
\newcommand{\cbJ}{\cb{J}}
\newcommand{\cbK}{\cb{K}}
\newcommand{\cbL}{\cb{L}}
\newcommand{\cbM}{\cb{M}}
\newcommand{\cbN}{\cb{N}}
\newcommand{\cbO}{\cb{O}}
\newcommand{\cbP}{\cb{P}}
\newcommand{\cbQ}{\cb{Q}}
\newcommand{\cbR}{\cb{R}}
\newcommand{\cbS}{\cb{S}}
\newcommand{\cbT}{\cb{T}}
\newcommand{\cbU}{\cb{U}}
\newcommand{\cbV}{\cb{V}}
\newcommand{\cbW}{\cb{W}}
\newcommand{\cbX}{\cb{X}}
\newcommand{\cbY}{\cb{Y}}
\newcommand{\cbZ}{\cb{Z}}

%%%%%%%%%%%%%%%%%%%%%%%%%%%%%%%%%%%%%%%%%%%%%%%%%%%%%%%%%%%%%%%%%%%%%%%%
% sets
%%%%%%%%%%%%%%%%%%%%%%%%%%%%%%%%%%%%%%%%%%%%%%%%%%%%%%%%%%%%%%%%%%%%%%%%
\newcommand{\set}[1]{\mathcal{#1}}

%% suffix
\newcommand{\Gc}{\set{G}}

%% prefix
\newcommand{\cA}{\set{A}}
\newcommand{\cB}{\set{B}}
\newcommand{\cC}{\set{C}}
\newcommand{\cD}{\set{D}}
\newcommand{\cE}{\set{E}}
\newcommand{\cF}{\set{F}}
\newcommand{\cG}{\set{G}}
\newcommand{\cH}{\set{H}}
\newcommand{\cI}{\set{I}}
\newcommand{\cJ}{\set{J}}
\newcommand{\cK}{\set{K}}
\newcommand{\cL}{\set{L}}
\newcommand{\cM}{\set{M}}
\newcommand{\cN}{\set{N}}
\newcommand{\cO}{\set{O}}
\newcommand{\cP}{\set{P}}
\newcommand{\cQ}{\set{Q}}
\newcommand{\cR}{\set{R}}
\newcommand{\cS}{\set{S}}
\newcommand{\cT}{\set{T}}
\newcommand{\cU}{\set{U}}
\newcommand{\cV}{\set{V}}
\newcommand{\cW}{\set{W}}
\newcommand{\cX}{\set{X}}
\newcommand{\cY}{\set{Y}}
\newcommand{\cZ}{\set{Z}}

%%%%%%%%%%%%%%%%%%%%%%%%%%%%%%%%%%%%%%%%%%%%%%%%%%%%%%%%%%%%%%%%%%%%%%%%
% operators
%%%%%%%%%%%%%%%%%%%%%%%%%%%%%%%%%%%%%%%%%%%%%%%%%%%%%%%%%%%%%%%%%%%%%%%%
\newcommand{\op}[1]{\operatorname{\text{#1}}}
\newcommand{\cddot}{\op{:}}
\newcommand{\tr}{\op{tr}}
\renewcommand{\div}{\op{div}}
\newcommand{\grad}{\op{grad}}
\newcommand{\curl}{\op{curl}}
\newcommand{\Div}{\op{Div}}
\newcommand{\Grad}{\op{Grad}}
\newcommand{\Curl}{\op{Curl}}
\newcommand{\vol}{\op{vol}}
\newcommand{\dev}{\op{dev}}
\newcommand{\Vol}{\op{Vol}}
\newcommand{\Dev}{\op{Dev}}
\newcommand*\diff{\mathop{}\!\mathrm{d}}


\begin{document}

\setlength{\headheight}{15pt}
\headsep = 4pt
%\pagestyle{fancyplain}

%\lhead
%[\fancyplain{}{\emph{Authors}}]
%{\fancyplain{}{\emph{Authors}}}

%\rhead
%[\fancyplain{} {\emph{Gurson model small deformation}}]
%{\fancyplain{} {\emph{Gurson model small deformation}}}

%Remove this in final version
%\cfoot
%[\fancyplain{}{\bf{DRAFT of \today}}]
%{\fancyplain{}{\bf{DRAFT of \today}}}

\title{On the formulation and numerical integration of a shear-modified Gurson model for ductile failure simulation at finite strain}

    \author{
     Author Information \\
	 }

\date{\today}

\maketitle

%%%%%%%%%%%%%%%%%%%%%%%%%%%%%%%%%%%%%%%
% INTRODUCTION
%%%%%%%%%%%%%%%%%%%%%%%%%%%%%%%%%%%%%%%
\section{Introduction} \label{sec:introduction}
\textit{TODO: add literature review}



%%%%%%%%%%%%%%%%%%%%%%%%%%%%%%%%%%%%%%%
% PRELIMINARIES
%%%%%%%%%%%%%%%%%%%%%%%%%%%%%%%%%%%%%%%
\section{Preliminaries for finite deformation hyperelastic formulation}\label{sec:preliminaries}

%----------------
% Kinematic
%----------------
\subsection{Kinematic preliminaries}
To set the stage for the hyperelastic formulation, the kinematic
preliminaries for large deformation elastoplasticity are summarized 
in this section, along with the quantities and relations that will be
used in subsequent model development.

An essential feature of this elastoplasticity framework is the
multiplicative decomposition of the deformation gradient $\bF$ into 
an elastic part $\bF^e$ and a plastic part $\bF^p$

\begin{equation}
  \bF = \bF^e \cdot \bF^p.
\end{equation}

This decomposition introduces the notion of an intermediate local
configuration (\cf\ \cite{SimoHughes:98} and the references therein
for the motivation and micromechanical basis for such a
decomposition).

Next, we introduce a set of strain measures associated with the
multiplicative decomposition that will be used extensively in the
model development. First is the right Cauchy-Green tensor $\bC$, and
its plastic counterpart $\bC^p$, which are defined in the reference
configuration

\begin{align}
  \bC &= \bF^T \cdot \bF\\ \bC^p &= \bF^{pT} \cdot \bF^p
\end{align}
where $\bF^T$ is the transpose of $\bF$.

In the current configuration we consider the left Cauchy-Green tensor
$\bb$, and its elastic counterpart ${\bb}^e$

\begin{align}
  \bb &= \bF \cdot \bF^{T}\\ \bb^e &= \bF^e \cdot
  \bF^{eT}\label{eq:be}
\end{align}

The above fundamental strain measures are related via pull-back and
push-forward operations

\begin{align}\label{eq:pullpush}
  \bC^{-p} &=\bF^{-1} \cdot \bb^e \cdot \bF^{-T}\\ \bb^e &= \bF \cdot
  \bC^{-p} \cdot \bF^T
\end{align}

In metal plasticity, a standard assumption is that plastic flow is
isochoric (volume-preserving), \ie\ $\det(\bF^p)=1$, which implies

\begin{equation}
  J = \det (\bF) = \det (\bF^e)
\end{equation}

Defining $J^e=\det(\bF^e)$, we have the volume preserving part of the
elastic left Cauchy-Green tensor $\bar{\bb}^e$

\begin{equation}
  \bar{\bb}^e = J^{e^{-2/3}}\bb^e = J^{-2/3}\bb^e.
\end{equation}

%----------------
% Hyperelastic
%----------------
\subsection{Hyperelastic constitutive relation}
The starting point of the hyperelastic constitutive formulation is 
the assumption of the existence of a strain-energy function, which is
proposed to have the following form

\begin{equation}
\Psi=\Psi^{\rm vol}[J^e] + \Psi^{\rm iso}[\bar{\bb}^e]
\end{equation}

Here the strain-energy function $\Psi$ is a decoupled function of the
volumetric part (i.e., $J^e=\det\bF^e$) and the isochoric part (i.e.,
$\bar{\bb}^e$) of the elastic deformation. The volumetric and the
isochoric parts of the strain-energy function are given as

\begin{equation}
  \Psi^{\rm vol}[J^e] =\frac{\kappa}{4}\left(J^e{^2} - 1 - \rm ln~J^e{^2}\right)
\end{equation}

\begin{equation}
  \Psi^{\rm iso}[\bar{\bb}^e] =
  \frac{\mu}{2}\left(\rm tr(\bar{\bb}^e)-3\right)
\end{equation}

where $\kappa$ and $\mu$ are the bulk and shear modulus. Here we are
following the developments of (\cite{Steinmann1994}), however we
employ a different choice of strain energy-functions. The elastic
constitutive law and the Kirchhoff stresses are given as

\begin{equation}\label{eq:Kirchhoff}
\btau=J^e p \bone + \bs
\end{equation}

The Kirchhoff pressure $p$ and the deviatoric stress tensor $\bs$ are
related to the elastic strain measure as

\begin{equation}\label{eq:p}
p = \frac{1}{3}\tr(\btau) = \frac{\kappa}{2}\left(J^e{^2}-1\right)/J^e
\end{equation}

\begin{equation}\label{eq:s}
\bs = \dev(\btau) = \mu\dev(\bar{\bb}^e)
\end{equation}

%%%%%%%%%%%%%%%%%%%%%%%%%%%%%%%%%%%%%%%
% FORMULATION
%%%%%%%%%%%%%%%%%%%%%%%%%%%%%%%%%%%%%%%
\section{Formulation of the finite deformation Gurson model} \label{sec:formulation}
Within the previously described large deformation hyperelastic
framework, key components of the constitutive relations of the
shear-modified Gurson model are presented in this section, including
the yield function, the hardening law, the flow rule and the 
evolution law for the shear-modified void growth.

%----------------
% Yield function
%----------------
\subsection{Yield function}
The yield function $\Phi$ of the Gurson model can be written in terms
of the previously defined Kirchhoff mean stress $p$ and deviatoric
stress tensor $\bs$ as

\begin{equation}\label{eq:Phi}
\Phi = \frac{1}{2}\bs:\bs - \frac{1}{3}\psi Y^2
\end{equation}

where $\psi$ contains contributions from the damage in the material
and $Y$ is the Kirchhoff yield stress. This yield function
\eqref{eq:Phi} is a quadratic version form. This particular variant 
of the yield function was chosen as it proved to exhibit superior
properties of robustness in convergence in the nonlinear solution
algorithm outline below.

The function $\psi$ directly relates to the void volume fraction of
the porous solid and is given as

\begin{equation}
\psi = 1 + q_3 f^2 - 2 q_1 f \cosh(v), ~~v=\frac{3 q_2 p}{2 Y}
\end{equation}

where $p$ is the Kirchhoff pressure defined in \eqref{eq:p}, and $f$
is the void volume fraction of the porous solid. The Kirchhoff yield
stress $Y$ describes the hardening of the undamaged matrix
material. Also, the constants $q_1,q_22,q_3$ are the accepted
additional yield parameters often present in GTN type models.

%----------------
% Hardening law
%----------------
\subsection{Hardening law}
The hardening law relates the yield strength $Y$ to some measure of
plastic deformation. One example of a nonlinear hardening law proposed
by \cite{SimoHughes:98} for metal is written as
\begin{equation}
Y = Y_0 + Y_{\infty}\left[ 1-\exp(-\delta\varepsilon_q)\right] +
K\varepsilon_q
\end{equation}

where $\varepsilon_q$ is the equivalent plastic strain, $Y_0$ is the
initial yield strength, $Y_{\infty}$ is the residual flow stress, $K$
is the hardening coefficient, and $\delta$ is the saturation
exponent. Other forms of hardening law can also be used depending on
the observed material behavior.

The plastic work increment in the matrix material is equal to the
macroscopic plastic work increment, which can be used to derive the
evolution equation for the equivalent plastic strain $\varepsilon_q$
as

\begin{equation}\label{eq:doteq_1}
\dot{\varepsilon}_q Y(1-f) = \btau : \left(
\gamma\frac{\partial\Phi}{\partial\btau} \right)
\end{equation}

Substituting yield function $\Phi$ into \eqref{eq:doteq_1} to obtain
the expression to compute $\dot{\varepsilon}_q$
\begin{equation}\label{eq:doteq}
  \dot{\varepsilon}_q =\frac{1}{1-f}\left(
  \gamma\sqrt{\frac{2|\psi|}{3}} {\rm sign}(\psi) + \frac{p t}{Y}
  \right)
\end{equation}


%----------------
% Flow rule
%----------------
\subsection{Flow rule}
Following the standard procedure of the principle of maximum
dissipation, Simo and Miehe \cite{Simo1992} proposed a general form 
of associate flow rule, which is adopted in the current formulation 
and is given as

\begin{equation}\label{eq:flowRule}
-\frac{1}{2}L_v(\bb^e)\cdot\bb^{-e} =
\gamma\frac{\partial\Phi}{\partial\btau}=\gamma\bn +
\frac{1}{3}t\tensor 1
\end{equation}

where $L_v(\bb^e)=\bF \cdot \dot{\bC}^{-p} \cdot \bF^T$ is the Lie
derivative, and $\bn$ and $t$ are the deviatoric and volumetric
component of the gradient term, respectively. Substituting the yield
function \eqref{eq:Phi} leads to the following

\begin{equation}\label{eq:n}
\bn = \frac{\bs}{\|\bs\|}
\end{equation}

\begin{equation}\label{eq:t}
t = \tr \left( \gamma \frac{\partial\Phi}{\partial\btau}\right) =
\sqrt{\frac{3}{2}}\gamma f \sinh (\frac{3p}{2
  Y})\frac{1}{\sqrt{|\psi|}} {\rm sign}(\psi)
\end{equation}


%----------------
% Void volume
%----------------
\subsection{Evolution of void volume fraction and shear modification}
The void volume fraction $f$ is the internal variable that
characterizes the material damage. The rate of change in total void
volume fraction, $\dot{f}$, is typically given by the sum of
contributions due to the void growth, $\dot{f}_g$, and the nucleation
of new voids, $\dot{f}_{n}$.

\begin{equation}\label{eq:dotf}
\dot{f} = \dot{f}_g + \dot{f}_{n}\\
\end{equation}

In the original Gurson model \cite{Gurson1977}, the void growth part
$\dot{f}_g$ was related to the plastic volume change as

\begin{equation}\label{eq:fg_orig}
\dot{f}_g = (1-f)\tr\left( \gamma\frac{\partial\Phi}{\partial\btau}
\right)
\end{equation}

To account for the void growth under shear-dominated stress state, 
the void growth law \eqref{eq:fg_orig} was extended in
\cite{Nahshon2008} by adding a term that depends on the third stress 
invariant. This shear-modified void growth equation is written as

\begin{equation}\label{eq:fg_1}
\dot{f}_g = (1-f)\tr\left( \gamma\frac{\partial\Phi}{\partial\btau}
\right) + k_{\omega}f\frac{\omega(\btau)}{\tau_e}\bs: \left(
\gamma\frac{\partial\Phi}{\partial\btau} \right)
\end{equation}

where $\tau_e=\sqrt{3/2}\|\bs\|$ is the effective deviatoric 
Kirchhoff stress, $k_{\omega}$ is a material constant that sets the 
magnitude of the damage growth rate in pure shear states 
\cite{Nahshon2008}. The function $\omega(\btau)$ includes the effect 
of the third stress invariant on void growth and is given as

\begin{equation}\label{eq:omega}
\omega(\btau) = 1- \left(\frac{27J_3}{2\tau_e^3}\right)^2
\end{equation}
where $J_3=\det(\bs)$ is the third invariant of deviatoric Kirchhoff
stress tensor.

Substituting the yield function \eqref{eq:Phi} and the expression for
$\omega(\btau)$ \eqref{eq:omega} into the void growth law
\eqref{eq:fg_1} yields

\begin{equation}\label{eq:fg_final}
\dot{f}_g = (1-f)t + \sqrt{\frac{2}{3}}\gamma k_{\omega}f
\omega(\btau)
\end{equation}

The effective increase in damage due to plastic strain controlled
nucleation is given by \cite{Chu1980} as

\begin{equation}\label{eq:dotf_nu}
\dot{f}_{nu} = A \dot{\varepsilon}_q
\end{equation}

where the parameter A is defined as a function of the matrix
equivalent plastic strain $\varepsilon_q$

\begin{equation}
A =
   \begin{cases}
     \frac{f_N}{s_N\sqrt{2\pi}}\exp\left[ -\frac{1}{2}\left(
       \frac{\varepsilon_q - \epsilon_N}{s_N}\right)^2\right], & p
     \geq 0\\ 0, & p <0
	\end{cases}
\end{equation}

where the nucleation strain follows a normal distribution with a mean
value $\epsilon_N$ and a standard deviation $s_N$ with the volume
fraction of the nucleated voids given by $f_N$.


%%%%%%%%%%%%%%%%%%%%%%%%%%%%%%%%%%%%%%%
% INTEGRATION
%%%%%%%%%%%%%%%%%%%%%%%%%%%%%%%%%%%%%%%
\section{Integration of constitutive equations} \label{sec:integration}

%----------------
% Discrete form
%----------------
\subsection{Discrete form of the rate equations}
To derive the discrete form of the rate equations, the starting point 
is to write the evolution equations in the material (reference) 
configuration. For the flow rule, a pull-back operation is applied to 
\eqref{eq:flowRule} such that
%
\begin{equation}\label{eq:flowRule_ref}
-\frac{1}{2}\dot{\bC}^{-p}\cdot\bC^p 
= \gamma\bF^{-1} \cdot \frac{\partial\Phi}{\partial\btau} \cdot \bF
\end{equation}
%
where $\gamma$ is the plastic multiplier. Then, the application of the 
exponential mapping to \eqref{eq:flowRule_ref} yields an incremental 
objective integration algorithm
%
\begin{equation}\label{eq:Cp}
\bC_{n+1}^{-p} = \bF_{n+1}^{-1} \cdot \exp\left( -2\Delta\gamma
\frac{\partial\Phi_{n+1}}{\partial\btau}\right) \cdot \bF_{n+1} 
\cdot \bC_{n}^{-p}
\end{equation}
%
Applying the push-forward operation to \eqref{eq:Cp} yields an update
algorithm for the elastic left Cauchy-Green tensor $\bb_{n+1}^e$ as
%
\begin{equation}\label{eq:be_n}
\bb_{n+1}^e = \exp\left( -2\Delta\gamma\frac{\partial\Phi_{n+1}}
{\partial\btau}\right)\cdot\bb^{e\tr}
\end{equation}
%
where the trial elastic left Cauchy-Green tensor $\bb^{e\tr}$ is given 
by
%
\begin{equation}\label{eq:betr}
\bb^{e\tr} = \bF_{n+1}\cdot\bC_{n}^{-p}\cdot\bF^T_{n+1}
\end{equation}
%
From elastic and plastic isotropy, $\bb_{n+1}^e, \bb^{e\tr}$ and 
$\btau$ have identical principal axes. Then, the logarithmic Hencky
strains follow as
%
\begin{equation}\label{eq:lnbetr}
\ln\bb_{n+1}^e = \ln\bb^{e\tr} - 
2\Delta\gamma\frac{\partial\Phi_{n+1}}{\partial\btau}
\end{equation}
%
Using the elastic constitutive relations \eqref{eq:p} and 
\eqref{eq:s}, the Kirchhoff pressure and deviatoric stress tensor at 
time $t_{n+1}$ can be obtained as
%
\begin{equation}\label{eq:discrete_p}
p_{n+1} = p^{\tr} - \kappa t
\end{equation}
%
\begin{equation}\label{eq:discrete_s}
\bs_{n+1} = \bs^{\tr} - 2\mu\Delta\gamma \bn
\end{equation}
%
where $\bn$ and $t$ are given by \eqref{eq:n}, \eqref{eq:t} and 
are evaluated at time $t_{n+1}$, and $p^{\tr}$ and $\bs^{\tr}$ are the 
trial states given by
%
\begin{equation}\label{eq:ptr}
p^{\tr} = \kappa\ln J^{e\tr},~~J^{e\tr} = \det(\bb^{e\tr})^{1/2}
\end{equation}
%
\begin{equation}\label{eq:str}
\bs^{\tr} = \mu\dev\ln\bb^{e\tr}
\end{equation}
%
The discrete form of evolution equations for internal variable 
$\varepsilon_q$ and the void volume fraction $f$ are obtained by apply 
backward Euler scheme to their evolution equations \eqref{eq:dotf} and 
\eqref{eq:doteq}. The resulting discrete equations are
%
\begin{equation}\label{eq:discrete_f}
f_{n+1} = f_n + (1-f_{n+1})t + \sqrt{\frac{2}{3}}\Delta\gamma
k_{\omega}f_{n+1} \omega(\btau) + A_{n+1}(\varepsilon_{q(n+1)} - 
\varepsilon_{q(n)})
\end{equation}
%
\begin{equation}\label{eq:discrete_eq}
\varepsilon_{q(n+1)} = \varepsilon_{q(n)} + \frac{1}{1-f_{n+1}}
\left(\Delta\gamma\sqrt{\frac{2|\psi|}{3}} {\rm sign}(\psi) + 
\frac{p_{n+1} t}{Y}\right)
\end{equation}


%----------------
% Nonlinear
%----------------
\subsection{Local nonlinear system of equations}
The discrete form of the rate equations \eqref{eq:be_n}, 
\eqref{eq:discrete_p}, \eqref{eq:discrete_s}, \eqref{eq:discrete_f} 
and \eqref{eq:discrete_eq} include four unknowns quantities relate to 
the stresses and internal state variables at time $t_{n+1}$, which 
are the pressure $p$, the equivalent plastic strain $\varepsilon_q$, 
the void volume fraction $f$, and the plastic multiplier $\Delta
\gamma$.

The unknowns will be obtained from solving the following nonlinear 
system of equations. For simplicity, in the following, we will omit 
the index $n+1$ referring to the current time $t_{n+1}$. The 
resulting nonlinear system of equations are
%
\begin{align}
R_1(\bX) &= \|\bs^{\tr}\| - 2\mu\Delta\gamma - 
\sqrt{\frac{2}{3}}{\rm sign}(\psi)\sqrt{|\psi|}Y\label{eq:R1}\\
R_2(\bX) &= p - p^{\tr} + \kappa t\label{eq:R2}\\
R_3(\bX) &= f - f_n - (1-f) t - \sqrt{\frac{2}{3}}\Delta\gamma 
k_{\omega}f \omega(\btau) - A(\varepsilon_{q} - \varepsilon_{q(n)})
\label{eq:R3}\\
R_4(\bX) &=\varepsilon_q - \varepsilon_{q(n)}- \frac{1}{1-f}
\left(\Delta\gamma\sqrt{\frac{2|\psi|}{3}} {\rm sign}(\psi) + 
\frac{p t}{Y}\right)\label{eq:R4}
\end{align}
%
where, the vector of unknowns $\bX$ is
%
\begin{equation}\label{eq:X}
\bX=\lbrace p, f, \varepsilon_q, \Delta\gamma \rbrace
\end{equation}
%
The above nonlinear system of equations can be solved through 
iterative solution method such as the Newton's method. The implicit 
algorithm for integrating the shear-modified large deformation Gurson 
model is summarized in the following box.\\

Box 1. Implicit algorithm for integrating shear-modified large 
deformation Gurson model\\

\fbox{\parbox{14.5cm}{
GIVEN: $\varepsilon_{q(n)}, f_n, \bb^e_n$ and $\bF$\\
FIND: $~~~\btau, \varepsilon_{q}, f, \bb^e (\text{or}~\bF^{p}) $ at 
time $t_{n+1}$\\
STEP 1. Compute trial elastic left Cauchy-Green tensor $\bb_e^{\tr}$ 
\eqref{eq:betr}\\
STEP 2. Compute trial stresses $p^{\tr}, \bs^{\tr}$ \eqref{eq:ptr},
\eqref{eq:str}\\
STEP 3. Check yielding \eqref{eq:Phi}:
$\Phi^{\tr}(p^{\tr}, \bs^{\tr}, \varepsilon_{q(n)}, f_n) > 0$ ?\\
$~~~~~~~~~~~~~~$No, set $p = p^{\tr}, \bs = \bs^{\tr}, \bb^e = 
\bb^{e\tr}, \varepsilon_{q} = \varepsilon_{q(n)}, f = f_n$  and 
exit\\
STEP 4. Yes, local Newton loop\\
$~~~~~~~~~~~~~~$4.1 Initialize $\bX^{k}$ \eqref{eq:X} and the 
iteration count $k=0$\\
$~~~~~~~~~~~~~~$4.2 Assemble the residual equations $\bR(\bX^k)$ 
\eqref{eq:R1} - \eqref{eq:R4}\\
$~~~~~~~~~~~~~~$4.3 Check convergence: $\parallel \bR \parallel < tolerance$ ?\\
$~~~~~~~~~~~~~~~~~~~$Yes, converged and go to STEP 5\\
$~~~~~~~~~~~~~~$4.4 No, compute local Jacobian matrix 
$\bJ = \partial\bR / \partial\bX$\\
$~~~~~~~~~~~~~~$4.5 Solve system of equations 
$\bJ \cdot \delta\bX = \bR$ for $\delta\bX$\\
$~~~~~~~~~~~~~~$4.6 Update $\bX^{k+1} = \bX^{k} - \delta\bX$, 
$k\rightarrow k+1$ and go to 4.2\\
STEP 5. Update $\btau = \bs + p\bg$, $\varepsilon_q, f$, and 
$\bF^{p}$\\
}}

The plastic deformation gradient $\bF^p$, which is used in 
\eqref{eq:flowRule_ref} and \eqref{eq:betr} to compute trial state, is 
updated using
%
\begin{equation}
\bF^p = \exp\left( \frac{\partial\Phi}{\partial\btau}\right)
\cdot\bF_{n}^p
\end{equation}
%
In the Newton's iterative solution method, it requires consistent 
linearization of the system of equation \eqref{eq:R1} - \eqref{eq:R4},
which necessities a derivative of the objective functions with respect 
to the independent fields (i.e., the unknowns). The Jacobian 
derivative of the objective function 
($\bJ = \partial\bR / \partial\bX$) is commonly referred to as the 
algorithmic consistent tangent operator in the constitutive model 
literature \cite{Miehe1996, Simo1985}. In this work, a technique in 
computational science called the forward automatic differentiation 
(FAD) will be used to compute necessary derivatives. FAD provides an 
efficient and convenient way to evaluate derivatives. It will be 
detailed in the next section. Interested readers can also refer to 
\cite{Chen2014} for applications of FAD to constitutive modeling in 
small- and large-deformation computational inelasticity.


%%%%%%%%%%%%%%%%%%%%%%%%%%%%%%%%%%%%%%%
% IMPLEMENTATION
%%%%%%%%%%%%%%%%%%%%%%%%%%%%%%%%%%%%%%%
\section{Numerical implementation and solution procedure} \label{sec:implementation}

%----------------
% Albany
%----------------
\subsection{Albany analysis code and solution procedure}
\textit{TODO: ref to our Journal of Nuclear Material paper. Include a 
description of the Albany analysis code and the solution procedure.}


%----------------
% FAD
%----------------
\subsection{FAD: a numerical exact way of computing consistent tangent}                                                                     
\textit{Need to modify to include more details of FAD}

The FAD technique is applied towards computing the tangent operator ($
\bJ = \partial\bR / \partial\bX$) which involves first- and second-
derivatives of the local system of residual equations \eqref{eq:R1}-
\eqref{eq:R4}, with respect to the unknown vector \eqref{eq:X}. The 
implementation is presented in Sandia National Laboratories' Albany 
analysis code\cite{Salinger2013}, which utilizes the Sacado package 
contained in Sandia's Trilinos framework to supply the 
automatic differentiation capabilities employed. To utilize the FAD 
technique for computing the local tangent operator, one must template 
both the system of residual equations and unknown vectors in terms of 
a Sacado FAD type data instead of the typical double precision data 
type. The unknown {\em state vector} will be the independent 
variable,  while the residual equations will be generic functions 
dependent on  the state vectors. The FAD data type contains not only 
the value of the data but also the derivative of the data with 
respect to the independent variables. The derivative information is 
initialized appropriately and propagated forward through the 
algorithm. In this way, once the sensitivities are initialized, the 
Jacobian or tangent operator will be calculated automatically. AD is 
also employed in the Albany analysis code to form the global system 
Jacobian, or stiffness matrix, for solving boundary value problems.


%%%%%%%%%%%%%%%%%%%%%%%%%%%%%%%%%%%%%%%
% VERIFICATION
%%%%%%%%%%%%%%%%%%%%%%%%%%%%%%%%%%%%%%%
\section{Verification - benchmark element tests} \label{sec:verification}
\textit{TODO: add benchmark element test example}


%%%%%%%%%%%%%%%%%%%%%%%%%%%%%%%%%%%%%%%
% Ductile failure
%%%%%%%%%%%%%%%%%%%%%%%%%%%%%%%%%%%%%%%
\section{Ductile failure simulation} \label{sec:ductile}

%----------------
% Experiment
%----------------
\subsection{Experimental program and results}
\textit{TODO: a brief description of the experiment program.}



%----------------
% Simulation
%----------------
\subsection{Numerical simulations}



%%%%%%%%%%%%%%%%%%%%%%%%%%%%%%%%%%%%%%%
% CONCLUSIONS
%%%%%%%%%%%%%%%%%%%%%%%%%%%%%%%%%%%%%%%
\section{Conclusions} \label{sec:conclusions}



\bibliographystyle{plainnat}
\bibliography{GursonJournal}

\end{document}