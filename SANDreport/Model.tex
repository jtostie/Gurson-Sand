\chapter{Model Formulation}
\label{model-form}

\textit{This section should begin with a short description about the
  origins of the model, including the desire to capture porous solid
  behavior related to void growth with an inelastic body. Then some
  comments about the thermodynamic motivation to place it within a
  hyperelastic framework. It should include sections on the flow rule
  and each piece of the void volume fraction evolution equation.}

\section{Background}

The original Gurson model was developed by \cite{Gurson1977} based on rigorous micromechanical analysis of a characteristic volume element with spherical or cylindrical-shaped voids surrounded by rigid plastic matrix materials. This model was motivated by experimental observations of dilational plastic deformation during ductile fracture in porous metals, which generates considerable porosity due to the growth and nucleation of voids. Tvergaard and Needleman \cite{Tvergaard1984} later improved the original Gurson model by introducing additional parameters into the yield function and by introducing an effective void volume fraction term to account for coalescence of voids that better capture the damage growth rates. This model is known as the 'Gurson-Tvergaard-Needleman' or the GTN model.

One important limitation of the original Gurson or the GTN model is that the void growth (i.e., the damage) depends only on the mean stress. In a shear-dominated state, such as in a projectile penetration problem, the model is unable to predict damage growth if continuous void nucleation is not invoked. This limitation motivates a modification of the original void growth law to include a shear term proposed in \cite{Nahshon2008}. This modification, though phenomenological in nature, has been shown to improve the prediction of the Gurson model in situations where shear stress dominates \cite{Nahshon2008,Nahshon2009}. 

The original Gurson model and the recent shear modification were both formulated either with a small deformation assumption or within a hypoelastic framework. The large deformation typically encountered in a ductile failure simulation would render the small deformation formulation inappropriate. As for the hypoelastic formulation, there are well known drawbacks such as the non-zero work done in a closed cycle of elastic deformation, which violates the most important axiom of an elastic response (\cite{Belytschko2013}).

To avoid the above problems, in this section, the shear-modified Gurson model in \cite{Nahshon2008, Nahshon2009} will be reformulated within a large deformation hyperelastic constitutive framework. Since the elastic response is derived from a hyperelastic potential, the work done in a closed elastic deformation loop vanishes exactly. Furthermore, the hyperelastic formulation eliminates the need for incrementally objective stress update algorithms and can be easily integrated with frame-invariant formulations of anisotropic elasticity and anisotropic plastic yielding (\cite{Belytschko2013}).

\section{Preliminaries for large deformation hyperelastic formulation}

\subsection{Kinematic preliminaries}
To set the stage for the hyperelastic formulation, the kinematic
preliminaries for large deformation elastoplasticity are summarized
in this section, along with the quantities and relations that will be used in
subsequent model development.

An essential feature of this elastoplasticity framework is the
multiplicative decomposition of the deformation gradient $\bF$ into an
elastic part $\bF^e$ and a plastic part $\bF^p$

\begin{equation}
  \bF = \bF^e \cdot \bF^p.
\end{equation}

This decomposition introduces the notion of an intermediate local
configuration (\cf\ \cite{SimoHughes:98} and the references
therein for the motivation and micromechanical basis for such a
decomposition).

Next, we introduce a set of strain measures associated with the
multiplicative decomposition that will be used extensively in the
model development. First is the right Cauchy-Green tensor $\bC$, and
its plastic counterpart $\bC^p$, which are defined in the reference configuration

\begin{align}
  \bC &= \bF^T \cdot \bF\\
  \bC^p &= \bF^{pT} \cdot \bF^p
\end{align}
where $\bF^T$ is the transpose of $\bF$.

In the current configuration we consider the left
Cauchy-Green tensor $\bb$, and its elastic counterpart ${\bb}^e$

\begin{align}
  \bb &= \bF \cdot \bF^{T}\\
  \bb^e &= \bF^e \cdot \bF^{eT}\label{eq:be}
\end{align}

The above fundamental strain measures are related via pull-back and
push-forward operations

\begin{align}\label{eq:pullpush}
  \bC^{-p} &=\bF^{-1} \cdot \bb^e \cdot \bF^{-T}\\
  \bb^e &= \bF \cdot \bC^{-p} \cdot \bF^T
\end{align}

In metal plasticity, a standard assumption is that plastic flow
is isochoric (volume-preserving), \ie\ $\det(\bF^p)=1$, which implies

\begin{equation}
  J = \det (\bF) = \det (\bF^e)
\end{equation}

Defining $J^e=\det(\bF^e)$, we have the volume preserving part of the
elastic left Cauchy-Green tensor $\bar{\bb}^e$

\begin{equation}
  \bar{\bb}^e = J^{e^{-2/3}}\bb^e = J^{-2/3}\bb^e.
\end{equation}

\subsection{Hyperelastic constitutive relation}

The starting point of the hyperelastic constitutive formulation is the assumption of the existence of a strain-energy function, which is proposed to have the following form

\begin{equation}
\Psi=\Psi^{\rm vol}[J^e] + \Psi^{\rm iso}[\bar{\bb}^e]
\end{equation}

Here the strain-energy function $\Psi$ is a decoupled function of the volumetric part (i.e., $J^e=\det\bF^e$) and the isochoric part (i.e., $\bar{\bb}^e$) of the elastic deformation. The volumetric and the isochoric parts of the strain-energy function are given as

\begin{equation}
  \Psi^{\rm vol}[J^e] =\frac{1}{2}\kappa(\ln J^e)^2
\end{equation}

\begin{equation}
  \Psi^{\rm iso}[\bar{\bb}^e] = \mu(\frac{1}{2}\ln\bar{\bb}^e):(\frac{1}{2}\ln\bar{\bb}^e)
\end{equation}

where $\kappa$ and $\mu$ are the bulk and shear modulus, and the elastic logarithmic Hencky strains $\frac{1}{2}\ln\bb^e$ is used as the strain measure (\cite{Steinmann1994}). The elastic constitutive law and the Kirchhoff stresses are given as

\begin{equation}\label{eq:Kirchhoff}
\btau=\kappa\ln J^e\bg^{-1} + \mu\ln\bar{\bb}^e = \kappa(\ln\bb^e:\bg)\bg^{-1} + \mu\dev\ln\bb^e
\end{equation}

where $\bg$ is the metric tensor. The Kirchhoff pressure $p$ and the deviatoric stress tensor $\bs$ are related to the elastic strain measure as

\begin{equation}\label{eq:p}
p = \frac{1}{3}\tr(\btau) = \frac{1}{2}\kappa\tr\ln\bb^e = \frac{1}{2}\kappa\ln\det\bb^e
\end{equation}

\begin{equation}\label{eq:s}
\bs = \dev(\btau) = \mu\dev\ln\bb^e
\end{equation}

\section{Constitutive relations of the Gurson model}
Within the previously described large deformation hyperelastic framework, key components of the constitutive relations of the shear-modified Gurson model are presented in this section, including the yield function, the hardening law, the flow rule and the evolution law for the shear-modified void growth.

\subsection{Yield function}

The yield function $\Phi$ of the Gurson model can be written in terms of the previously defined Kirchhoff mean stress $p$ and deviatoric stress tensor $\bs$ as

\begin{equation}\label{eq:Phi}
\Phi = \|\bs\| - \sqrt{\frac{2}{3}}{\rm sign}(\psi)\sqrt{|\psi|} Y
\end{equation}

where $\psi$ contains contributions from the damage in the material and $Y$ is the Kirchhoff yield stress. This yield function \eqref{eq:Phi} is a linear form, in terms of $\|\bs\|$, of the Gurson yield criteria. This particular form is implemented to facilitate comparisons with existing $J_2$-like models in the Albany analysis code \cite{Salinger2013}, where the yield functions are mostly written as linear functions of $\|\bs\|$.

The function $\psi$ directly relates to the void volume fraction of the porous solid and is given as

\begin{equation}
\psi = 1 + f^2 - 2 f \cosh(v), ~~v=\frac{3p}{2 Y}
\end{equation}

where $p$ is the Kirchhoff pressure defined in \eqref{eq:p}, and $f$ is the void volume fraction of the porous solid. The Kirchhoff yield stress $Y$ describes the hardening of the undamaged matrix material. 

\subsection{Hardening law}
The hardening law relates the yield strength $Y$ to some measure of plastic deformation. One example of a nonlinear hardening law proposed by \cite{SimoHughes:98} for metal is written as 
\begin{equation}
Y =  Y_0 + Y_{\infty}\left[ 1-\exp(-\delta\varepsilon_q)\right] + K\varepsilon_q 
\end{equation}

where $\varepsilon_q$ is the equivalent plastic strain, $Y_0$ is the initial yield strength, $Y_{\infty}$ is the residual flow stress, $K$ is the hardening coefficient, and $\delta$ is the saturation exponent. Other forms of hardening law can also be used depending on the observed material behavior.

The plastic work increment in the matrix material is equal to the macroscopic plastic work increment, which can be used to derive the evolution equation for the equivalent plastic strain $\varepsilon_q$ as

\begin{equation}\label{eq:doteq_1}
\dot{\varepsilon}_q Y(1-f) = \btau : \left( \gamma\frac{\partial\Phi}{\partial\btau} \right)
\end{equation}

Substituting yield function $\Phi$ into \eqref{eq:doteq_1} to obtain the expression to compute $\dot{\varepsilon}_q$
\begin{equation}\label{eq:doteq}
  \dot{\varepsilon}_q =\frac{1}{1-f}\left( \gamma\sqrt{\frac{2|\psi|}{3}} {\rm sign}(\psi) + \frac{p t}{Y} \right)
\end{equation}

\subsection{Flow rule}
Following the standard procedure of the principle of maximum dissipation, Simo and Miehe \cite{Simo1992} proposed a general form of associate flow rule, which is adopted in the current formulation and is given as

\begin{equation}\label{eq:flowRule}
-\frac{1}{2}L_v(\bb^e)\cdot\bb^{e^{-1}} = \gamma\frac{\partial\Phi}{\partial\btau}=\gamma\bn + \frac{1}{3}t\tensor 1
\end{equation}

where $L_v(\bb^e)=\bF \cdot \dot{\bC}^{p^{-1}} \cdot \bF^T$ is the Lie derivative, and $\bn$ and $t$ are the deviatoric and volumetric component of the gradient term, respectively. Substituting the yield function \eqref{eq:Phi} leads to the following

\begin{equation}\label{eq:n}
\bn = \frac{\bs}{\|\bs\|}
\end{equation}

\begin{equation}\label{eq:t}
t = \tr \left( \gamma \frac{\partial\Phi}{\partial\btau}\right) = \sqrt{\frac{3}{2}}\gamma f \sinh (\frac{3p}{2 Y})\frac{1}{\sqrt{|\psi|}} {\rm sign}(\psi)
\end{equation}

\subsection{Evolution of void volume fraction}
The void volume fraction $f$ is the internal variable that characterizes the material damage. The rate of change in total void volume fraction, $\dot{f}$, is typically given by the sum of contributions due to the void growth, $\dot{f}_g$, and the nucleation of new voids, $\dot{f}_{n}$.

\begin{equation}\label{eq:dotf}
\dot{f} = \dot{f}_g + \dot{f}_{n}\\
\end{equation}

In the original Gurson model \cite{Gurson1977}, the void growth part $\dot{f}_g$ was related to the plastic volume change as 

\begin{equation}\label{eq:fg_orig}
\dot{f}_g = (1-f)\tr\left( \gamma\frac{\partial\Phi}{\partial\btau} \right)
\end{equation}

To account for the void growth under shear-dominated stress state, the void growth law \eqref{eq:fg_orig} was extended in \cite{Nahshon2008} by adding a term that depends on the third stress invariant. This shear-modified void growth equation is written as

\begin{equation}\label{eq:fg_1}
\dot{f}_g = (1-f)\tr\left( \gamma\frac{\partial\Phi}{\partial\btau} \right) + k_{\omega}f\frac{\omega(\btau)}{\tau_e}\bs: \left( \gamma\frac{\partial\Phi}{\partial\btau} \right)
\end{equation}

where $\tau_e=\sqrt{3/2}\|\bs\|$ is the effective deviatoric Kirchhoff stress, $k_{\omega}$ is a material constant that sets the magnitude of the damage growth rate in pure shear states \cite{Nahshon2008}. The function $\omega(\btau)$ includes the effect of the third stress invariant on void growth and is given as

\begin{equation}\label{eq:omega}
\omega(\btau) = 1- \left(\frac{27J_3}{2\tau_e^3}\right)^2
\end{equation}
where $J_3=\det(\bs)$ is the third invariant of deviatoric Kirchhoff stress tensor.

Substituting the yield function \eqref{eq:Phi} and the expression for $\omega(\btau)$ \eqref{eq:omega} into the void growth law \eqref{eq:fg_1} yields

\begin{equation}\label{eq:fg_final}
\dot{f}_g = (1-f)t + \sqrt{\frac{2}{3}}\gamma k_{\omega}f \omega(\btau)
\end{equation}

The effective increase in damage due to plastic strain controlled nucleation is given by \cite{Chu1980} as

\begin{equation}\label{eq:dotf_nu}
\dot{f}_{nu} = A \dot{\varepsilon}_q
\end{equation}

where the parameter A is defined as a function of the matrix equivalent plastic strain $\varepsilon_q$

\begin{equation}
A = 
   \begin{cases}
     \frac{f_N}{s_N\sqrt{2\pi}}\exp\left[ -\frac{1}{2}\left( \frac{\varepsilon_q - \epsilon_N}{s_N}\right)^2\right], & p \geq 0\\
	 0,	  & p <0
	\end{cases}
\end{equation}

where the nucleation strain follows a normal distribution with a mean value $\epsilon_N$ and a standard deviation $s_N$ with the volume fraction of the nucleated voids given by $f_N$.

%\subsection{Summary of material parameters}
%The material parameters for the modified Gurson model include
%\begin{equation}
%\bP = \left\lbrace  E, \nu, K, Y_0, Y_{\infty}, \delta, f_0, k_{\omega}, e_N, s_N, f_N\right\rbrace 
%\end{equation}

% Local Variables:
% TeX-master: "GursonModel"
% mode: latex
% mode: flyspell
% End:
