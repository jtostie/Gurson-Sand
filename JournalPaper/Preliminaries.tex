\section{Preliminaries for finite deformation hyperelastic formulation}\label{sec:preliminaries}

%----------------
% Kinematic
%----------------
\subsection{Kinematic preliminaries}
To set the stage for the hyperelastic formulation, the kinematic
preliminaries for large deformation elastoplasticity are summarized 
in this section, along with the quantities and relations that will be
used in subsequent model development.

An essential feature of this elastoplasticity framework is the
multiplicative decomposition of the deformation gradient $\bF$ into 
an elastic part $\bF^e$ and a plastic part $\bF^p$

\begin{equation}
  \bF = \bF^e \cdot \bF^p.
\end{equation}

This decomposition introduces the notion of an intermediate local
configuration (\cf\ \cite{SimoHughes:98} and the references therein
for the motivation and micromechanical basis for such a
decomposition).

Next, we introduce a set of strain measures associated with the
multiplicative decomposition that will be used extensively in the
model development. First is the right Cauchy-Green tensor $\bC$, and
its plastic counterpart $\bC^p$, which are defined in the reference
configuration

\begin{align}
  \bC &= \bF^T \cdot \bF\\ \bC^p &= \bF^{pT} \cdot \bF^p
\end{align}
where $\bF^T$ is the transpose of $\bF$.

In the current configuration we consider the left Cauchy-Green tensor
$\bb$, and its elastic counterpart ${\bb}^e$

\begin{align}
  \bb &= \bF \cdot \bF^{T}\\ \bb^e &= \bF^e \cdot
  \bF^{eT}\label{eq:be}
\end{align}

The above fundamental strain measures are related via pull-back and
push-forward operations

\begin{align}\label{eq:pullpush}
  \bC^{-p} &=\bF^{-1} \cdot \bb^e \cdot \bF^{-T}\\ \bb^e &= \bF \cdot
  \bC^{-p} \cdot \bF^T
\end{align}

In metal plasticity, a standard assumption is that plastic flow is
isochoric (volume-preserving), \ie\ $\det(\bF^p)=1$, which implies

\begin{equation}
  J = \det (\bF) = \det (\bF^e)
\end{equation}

Defining $J^e=\det(\bF^e)$, we have the volume preserving part of the
elastic left Cauchy-Green tensor $\bar{\bb}^e$

\begin{equation}
  \bar{\bb}^e = J^{e^{-2/3}}\bb^e = J^{-2/3}\bb^e.
\end{equation}

%----------------
% Hyperelastic
%----------------
\subsection{Hyperelastic constitutive relation}
The starting point of the hyperelastic constitutive formulation is 
the assumption of the existence of a strain-energy function, which is
proposed to have the following form

\begin{equation}
\Psi=\Psi^{\rm vol}[J^e] + \Psi^{\rm iso}[\bar{\bb}^e]
\end{equation}

Here the strain-energy function $\Psi$ is a decoupled function of the
volumetric part (i.e., $J^e=\det\bF^e$) and the isochoric part (i.e.,
$\bar{\bb}^e$) of the elastic deformation. The volumetric and the
isochoric parts of the strain-energy function are given as

\begin{equation}
  \Psi^{\rm vol}[J^e] =\frac{\kappa}{4}\left(J^e{^2} - 1 - \rm ln~J^e{^2}\right)
\end{equation}

\begin{equation}
  \Psi^{\rm iso}[\bar{\bb}^e] =
  \frac{\mu}{2}\left(\rm tr(\bar{\bb}^e)-3\right)
\end{equation}

where $\kappa$ and $\mu$ are the bulk and shear modulus. Here we are
following the developments of (\cite{Steinmann1994}), however we
employ a different choice of strain energy-functions. The elastic
constitutive law and the Kirchhoff stresses are given as

\begin{equation}\label{eq:Kirchhoff}
\btau=J^e p \bone + \bs
\end{equation}

The Kirchhoff pressure $p$ and the deviatoric stress tensor $\bs$ are
related to the elastic strain measure as

\begin{equation}\label{eq:p}
p = \frac{1}{3}\tr(\btau) = \frac{\kappa}{2}\left(J^e{^2}-1\right)/J^e
\end{equation}

\begin{equation}\label{eq:s}
\bs = \dev(\btau) = \mu\dev(\bar{\bb}^e)
\end{equation}


% Local Variables:
% TeX-master: "GursonModel"
% mode: latex
% mode: flyspell
% End:
