\section{Integration of constitutive equations} \label{sec:integration}

%----------------
% Discrete form
%----------------
\subsection{Discrete form of the rate equations}
To derive the discrete form of the rate equations, the starting point 
is to write the evolution equations in the material (reference) 
configuration. For the flow rule, a pull-back operation is applied to 
\eqref{eq:flowRule} such that
%
\begin{equation}\label{eq:flowRule_ref}
-\frac{1}{2}\dot{\bC}^{-p}\cdot\bC^p 
= \gamma\bF^{-1} \cdot \frac{\partial\Phi}{\partial\btau} \cdot \bF
\end{equation}
%
where $\gamma$ is the plastic multiplier. Then, the application of the 
exponential mapping to \eqref{eq:flowRule_ref} yields an incremental 
objective integration algorithm
%
\begin{equation}\label{eq:Cp}
\bC_{n+1}^{-p} = \bF_{n+1}^{-1} \cdot \exp\left( -2\Delta\gamma
\frac{\partial\Phi_{n+1}}{\partial\btau}\right) \cdot \bF_{n+1} 
\cdot \bC_{n}^{-p}
\end{equation}
%
Applying the push-forward operation to \eqref{eq:Cp} yields an update
algorithm for the elastic left Cauchy-Green tensor $\bb_{n+1}^e$ as
%
\begin{equation}\label{eq:be_n}
\bb_{n+1}^e = \exp\left( -2\Delta\gamma\frac{\partial\Phi_{n+1}}
{\partial\btau}\right)\cdot\bb^{e\tr}
\end{equation}
%
where the trial elastic left Cauchy-Green tensor $\bb^{e\tr}$ is given 
by
%
\begin{equation}\label{eq:betr}
\bb^{e\tr} = \bF_{n+1}\cdot\bC_{n}^{-p}\cdot\bF^T_{n+1}
\end{equation}
%
From elastic and plastic isotropy, $\bb_{n+1}^e, \bb^{e\tr}$ and 
$\btau$ have identical principal axes. Then, the logarithmic Hencky
strains follow as
%
\begin{equation}\label{eq:lnbetr}
\ln\bb_{n+1}^e = \ln\bb^{e\tr} - 
2\Delta\gamma\frac{\partial\Phi_{n+1}}{\partial\btau}
\end{equation}
%
Using the elastic constitutive relations \eqref{eq:p} and 
\eqref{eq:s}, the Kirchhoff pressure and deviatoric stress tensor at 
time $t_{n+1}$ can be obtained as
%
\begin{equation}\label{eq:discrete_p}
p_{n+1} = p^{\tr} - \kappa t
\end{equation}
%
\begin{equation}\label{eq:discrete_s}
\bs_{n+1} = \bs^{\tr} - 2\mu\Delta\gamma \bn
\end{equation}
%
where $\bn$ and $t$ are given by \eqref{eq:n}, \eqref{eq:t} and 
are evaluated at time $t_{n+1}$, and $p^{\tr}$ and $\bs^{\tr}$ are the 
trial states given by
%
\begin{equation}\label{eq:ptr}
p^{\tr} = \kappa\ln J^{e\tr},~~J^{e\tr} = \det(\bb^{e\tr})^{1/2}
\end{equation}
%
\begin{equation}\label{eq:str}
\bs^{\tr} = \mu\dev\ln\bb^{e\tr}
\end{equation}
%
The discrete form of evolution equations for internal variable 
$\varepsilon_q$ and the void volume fraction $f$ are obtained by apply 
backward Euler scheme to their evolution equations \eqref{eq:dotf} and 
\eqref{eq:doteq}. The resulting discrete equations are
%
\begin{equation}\label{eq:discrete_f}
f_{n+1} = f_n + (1-f_{n+1})t + \sqrt{\frac{2}{3}}\Delta\gamma
k_{\omega}f_{n+1} \omega(\btau) + A_{n+1}(\varepsilon_{q(n+1)} - 
\varepsilon_{q(n)})
\end{equation}
%
\begin{equation}\label{eq:discrete_eq}
\varepsilon_{q(n+1)} = \varepsilon_{q(n)} + \frac{1}{1-f_{n+1}}
\left(\Delta\gamma\sqrt{\frac{2|\psi|}{3}} {\rm sign}(\psi) + 
\frac{p_{n+1} t}{Y}\right)
\end{equation}


%----------------
% Nonlinear
%----------------
\subsection{Local nonlinear system of equations}
The discrete form of the rate equations \eqref{eq:be_n}, 
\eqref{eq:discrete_p}, \eqref{eq:discrete_s}, \eqref{eq:discrete_f} 
and \eqref{eq:discrete_eq} include four unknowns quantities relate to 
the stresses and internal state variables at time $t_{n+1}$, which 
are the pressure $p$, the equivalent plastic strain $\varepsilon_q$, 
the void volume fraction $f$, and the plastic multiplier $\Delta
\gamma$.

The unknowns will be obtained from solving the following nonlinear 
system of equations. For simplicity, in the following, we will omit 
the index $n+1$ referring to the current time $t_{n+1}$. The 
resulting nonlinear system of equations are
%
\begin{align}
R_1(\bX) &= \|\bs^{\tr}\| - 2\mu\Delta\gamma - 
\sqrt{\frac{2}{3}}{\rm sign}(\psi)\sqrt{|\psi|}Y\label{eq:R1}\\
R_2(\bX) &= p - p^{\tr} + \kappa t\label{eq:R2}\\
R_3(\bX) &= f - f_n - (1-f) t - \sqrt{\frac{2}{3}}\Delta\gamma 
k_{\omega}f \omega(\btau) - A(\varepsilon_{q} - \varepsilon_{q(n)})
\label{eq:R3}\\
R_4(\bX) &=\varepsilon_q - \varepsilon_{q(n)}- \frac{1}{1-f}
\left(\Delta\gamma\sqrt{\frac{2|\psi|}{3}} {\rm sign}(\psi) + 
\frac{p t}{Y}\right)\label{eq:R4}
\end{align}
%
where, the vector of unknowns $\bX$ is
%
\begin{equation}\label{eq:X}
\bX=\lbrace p, f, \varepsilon_q, \Delta\gamma \rbrace
\end{equation}
%
The above nonlinear system of equations can be solved through 
iterative solution method such as the Newton's method. The implicit 
algorithm for integrating the shear-modified large deformation Gurson 
model is summarized in the following box.\\

Box 1. Implicit algorithm for integrating shear-modified large 
deformation Gurson model\\

\fbox{\parbox{14.5cm}{
GIVEN: $\varepsilon_{q(n)}, f_n, \bb^e_n$ and $\bF$\\
FIND: $~~~\btau, \varepsilon_{q}, f, \bb^e (\text{or}~\bF^{p}) $ at 
time $t_{n+1}$\\
STEP 1. Compute trial elastic left Cauchy-Green tensor $\bb_e^{\tr}$ 
\eqref{eq:betr}\\
STEP 2. Compute trial stresses $p^{\tr}, \bs^{\tr}$ \eqref{eq:ptr},
\eqref{eq:str}\\
STEP 3. Check yielding \eqref{eq:Phi}:
$\Phi^{\tr}(p^{\tr}, \bs^{\tr}, \varepsilon_{q(n)}, f_n) > 0$ ?\\
$~~~~~~~~~~~~~~$No, set $p = p^{\tr}, \bs = \bs^{\tr}, \bb^e = 
\bb^{e\tr}, \varepsilon_{q} = \varepsilon_{q(n)}, f = f_n$  and 
exit\\
STEP 4. Yes, local Newton loop\\
$~~~~~~~~~~~~~~$4.1 Initialize $\bX^{k}$ \eqref{eq:X} and the 
iteration count $k=0$\\
$~~~~~~~~~~~~~~$4.2 Assemble the residual equations $\bR(\bX^k)$ 
\eqref{eq:R1} - \eqref{eq:R4}\\
$~~~~~~~~~~~~~~$4.3 Check convergence: $\parallel \bR \parallel < tolerance$ ?\\
$~~~~~~~~~~~~~~~~~~~$Yes, converged and go to STEP 5\\
$~~~~~~~~~~~~~~$4.4 No, compute local Jacobian matrix 
$\bJ = \partial\bR / \partial\bX$\\
$~~~~~~~~~~~~~~$4.5 Solve system of equations 
$\bJ \cdot \delta\bX = \bR$ for $\delta\bX$\\
$~~~~~~~~~~~~~~$4.6 Update $\bX^{k+1} = \bX^{k} - \delta\bX$, 
$k\rightarrow k+1$ and go to 4.2\\
STEP 5. Update $\btau = \bs + p\bg$, $\varepsilon_q, f$, and 
$\bF^{p}$\\
}}

The plastic deformation gradient $\bF^p$, which is used in 
\eqref{eq:flowRule_ref} and \eqref{eq:betr} to compute trial state, is 
updated using
%
\begin{equation}
\bF^p = \exp\left( \frac{\partial\Phi}{\partial\btau}\right)
\cdot\bF_{n}^p
\end{equation}
%
In the Newton's iterative solution method, it requires consistent 
linearization of the system of equation \eqref{eq:R1} - \eqref{eq:R4},
which necessities a derivative of the objective functions with respect 
to the independent fields (i.e., the unknowns). The Jacobian 
derivative of the objective function 
($\bJ = \partial\bR / \partial\bX$) is commonly referred to as the 
algorithmic consistent tangent operator in the constitutive model 
literature \cite{Miehe1996, Simo1985}. In this work, a technique in 
computational science called the forward automatic differentiation 
(FAD) will be used to compute necessary derivatives. FAD provides an 
efficient and convenient way to evaluate derivatives. It will be 
detailed in the next section. Interested readers can also refer to 
\cite{Chen2014} for applications of FAD to constitutive modeling in 
small- and large-deformation computational inelasticity.


%%%%%%%%%%%%%%%%%%%%%%%%%%%%%%%%%%%%%%%%%%
% IMPLEMENTATION
%%%%%%%%%%%%%%%%%%%%%%%%%%%%%%%%%%%%%%%%%%
\section{Numerical implementation and solution procedure} \label{sec:implementation}

%----------------
% Albany
%----------------
\subsection{Albany analysis code and solution procedure}
\textit{TODO: ref to our Journal of Nuclear Material paper. Include a 
description of the Albany analysis code and the solution procedure.}


%----------------
% FAD
%----------------
\subsection{FAD: a numerical exact way of computing consistent tangent}                                                                     
\textit{Need to modify to include more details of FAD}

The FAD technique is applied towards computing the tangent operator ($
\bJ = \partial\bR / \partial\bX$) which involves first- and second-
derivatives of the local system of residual equations \eqref{eq:R1}-
\eqref{eq:R4}, with respect to the unknown vector \eqref{eq:X}. The 
implementation is presented in Sandia National Laboratories' Albany 
analysis code\cite{Salinger2013}, which utilizes the Sacado package 
contained in Sandia's Trilinos framework to supply the 
automatic differentiation capabilities employed. To utilize the FAD 
technique for computing the local tangent operator, one must template 
both the system of residual equations and unknown vectors in terms of 
a Sacado FAD type data instead of the typical double precision data 
type. The unknown {\em state vector} will be the independent 
variable,  while the residual equations will be generic functions 
dependent on  the state vectors. The FAD data type contains not only 
the value of the data but also the derivative of the data with 
respect to the independent variables. The derivative information is 
initialized appropriately and propagated forward through the 
algorithm. In this way, once the sensitivities are initialized, the 
Jacobian or tangent operator will be calculated automatically. AD is 
also employed in the Albany analysis code to form the global system 
Jacobian, or stiffness matrix, for solving boundary value problems.




% Local Variables:
% TeX-master: "GursonModel"
% mode: latex
% mode: flyspell
% End:
